\subsubsection{Parameters}

The parameters at the top allow the user to explore different settings:

Changing the blocking factor may decrease execution time if it enables
greater reuse of cached data.  On the other hand, increasing the
blocking factor by too great an amount could hurt performance, as there
is increased memory bandwidth consumed when pulling in the ghost cells
for more blocks.

Changing the cache available per core may affect what type of reuse
occurs in each loop of the program.  The spreadsheet compares the
available machine cache size with the working set sizes required for
reuse between pencil iterations and plane iterations.

The cost of reads (R), read-writes (RW), and writes (W) affect the
bandwidth consumed by each of the memory operations.  For example, in
a cache-bypass setting, the write bandwidth could be half that of the
read-write bandwidth, whereas if no cache-bypass is used, it could be
equal to the read-write bandwidth.

The last two parameters allows two optimizations to be made to how the
cache is managed.  If Streaming Writes is TRUE, it is assumed that data
from read-write and write-only arrays are streamed into registers and
do not pollute the cache.  If the additional optimization of using NTA
cache hints is used, it is assumed that streaming reads (reads with no
reuse) will not pollute the cache either.  These optimizations affect
the computed working set sizes required for different types of reuse
(reuse between plane sweeps and reuse between pencil sweeps).

\subsubsection{Loop Analysis Table}

This table lists the following properties for each loop in the code:

\begin{itemize}

\item Function name and line number of the loop

\item Number of sweeps (e.g. if it is run for each component in an array)

\item Total and block iteration space

\item Number of floating point operations per iteration (adds, multiplies,
and specials)

\item Number of arrays for read-write (RW) and write-only (W) access

\item Number of planes and pencils required in the working set to enable
reuse between sweeps

\item Four pairs of columns, each pair listing 1) the working set in MB
      to enable reuse between plane sweeps and 2) the working set in MB
      to enable reuse between pencil sweeps.  The first pair is for a
      naive memory access pattern.  The second pair lists working sets
      required if streaming writes are utilized.  The third pair lists
      the optimal working set where only data that is reused resides
      in cache.  The fourth pair lists the actual working sets based on
      the user's selections in the parameter table section.  If NTA Hints
      is TRUE, then the fourth pair of columns points to the third pair
      (reuse only).  If NTA Hints is FALSE, but Streaming Writes is TRUE,
      then the fourth pair points to the second pair (streaming writes).
      If both are FALSE, then the fourth pair points to the first pair
      (naive).

\item Four columns showing the memory bandwidth consumed under different
      scenarios.  The first column shows the amount of data transferred
      if there is reuse between plane sweeps of the cache block.
      This corresponds to the compulsory traffic for the cache block.
      The second column shows amount transferred if there is reuse between
      pencil sweeps, but not plane sweeps.  The third column shows the
      amount transferred if there is is no reuse between pencil sweeps.
      The fourth column shows the actual amount transferred given the
      size of the cache specified in the parameters table at the top of
      the spreadsheet and the working set sizes required for different
      types of reuse given in the ``Working set (actual)'' columns of
      this table.

\item Gflops performed per iteration and the arithmetic intensity of
      the computation given the Gflops performaned and the Gbytes
      transferred

\item The estimated execution times if the program is CPU limited (CPU),
      or memory bandwidth limited (DRAM).  The final estimate (CPU and
      DRAM) is the maximum between these two values.

\end{itemize}


